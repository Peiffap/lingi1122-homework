\documentclass{elsarticle}
\usepackage[a4paper,left=2.5cm,right=1.5cm,top=1.5cm,bottom=1.5cm]{geometry}
\usepackage{natbib}
\usepackage{amsmath,amssymb,amsfonts,amsthm}
\usepackage{mathtools}
\usepackage[french]{babel}
\usepackage{bm}
\usepackage{algorithmic}
\usepackage{graphicx}
\usepackage{textcomp}
\usepackage{xcolor}
\usepackage{hyperref}
\usepackage{float}
\usepackage[T1]{fontenc}
\usepackage[utf8]{inputenc}
\usepackage{subcaption}
\usepackage{listings}
\usepackage{../../dafny}
\lstset{language=dafny}
\graphicspath{{img/}}
\usepackage{svg}

\makeatletter
\def\ps@pprintTitle{%
	\let\@oddhead\@empty
	\let\@evenhead\@empty
	\def\@oddfoot{\centerline{\thepage}}%
	\let\@evenfoot\@oddfoot}
\makeatother

\makeatletter
\def\blfootnote{\gdef\@thefnmark{}\@footnotetext}
\makeatother

\def\BibTeX{{\rm B\kern-.05em{\sc i\kern-.025em b}\kern-.08em
		T\kern-.1667em\lower.7ex\hbox{E}\kern-.125emX}}
\usepackage{siunitx}

\newcommand{\abs}{\ensuremath{\textnormal{abs}_{\textnormal{Arbre 2-3}}}}

\newcommand{\ok}{\ensuremath{\textnormal{ok}_{\textnormal{Arbre 2-3}}}}

\newcommand{\seq}[1]{\big\langle #1 \big\rangle}

\DeclareMathOperator{\size}{size}
\DeclareMathOperator{\height}{height}
\DeclareMathOperator{\type}{type}
\DeclareMathOperator{\elements}{elements}

\renewcommand{\epsilon}{\varepsilon}
\renewcommand{\theta}{\vartheta}
\renewcommand{\kappa}{\varkappa}
\renewcommand{\rho}{\varrho}
\renewcommand{\phi}{\varphi}

\begin{document}
\title{Méthodes de conception de programmes \\ Devoir 2: 1, 2, 3\ldots{} Arbres!}
\date{30 avril 2019}

\address[add1]{École Polytechnique, Université catholique de Louvain, Place de l'Université 1, 1348 Ottignies-Louvain-la-Neuve, Belgique}

\author[add1]{Alexandre \textsc{Gobeaux}}
\ead{alexandre.gobeaux@student.uclouvain.be}

\author[add1]{Louis \textsc{Navarre}}
\ead{navarre.louis@student.uclouvain.be}

\author[add1]{Gilles \textsc{Peiffer}}
\ead{gilles.peiffer@student.uclouvain.be}

\begin{abstract}
Ce papier donne les invariants de représentation, la fonction d'abstraction et les spécifications des fonctions \inlinedafny|insert| et \inlinedafny|join| pour une implémentation des arbres 2-3 basée sur \cite{algs4th} et \cite{wiki:23tree}\footnote{La définition des arbres 2-3 de cette dernière source n'était pas valable lorsque l'arbre 2-3 contient des doublons, ce qui a été corrigé.}.
\end{abstract}
\maketitle

\section{Invariant de représentation}
Commençons par définir quelques fonctions auxiliaires:
\begin{itemize}
	\item \(\size(t)\): donne le nombre de n\oe{}uds d'un arbre \(t\) (on considère ici que \inlinedafny|Vide| est un n\oe{}ud de taille \(0\));
	\item \(\height(t)\): donne la hauteur d'un arbre \(t\);
	\item \(\type(t)\): donne le nombre de sous-arbres du n\oe{}ud source de l'arbre \(t\).
\end{itemize}
Afin d'alléger la notation de l'invariant de représentation \(\ok(t)\), voici quelques conventions supplémentaires.
\begin{itemize}
	\item Si le n\oe{}ud source de \(t\) est un 2-n\oe{}ud, alors \(\ell\) et \(r\) dénotent respectivement le sous-arbre de gauche et de droite de \(t\), alors que \(a\) dénote la donnée de son n\oe{}ud source.
	\item Si le n\oe{}ud source de \(t\) est un 3-n\oe{}ud, alors \(\ell\), \(m\) et \(r\) dénotent respectivement le sous-arbre de gauche, du milieu et de droite de \(t\), alors que \(a \le b\) sont les données du n\oe{}ud source.
\end{itemize}
\begin{align}
f(t) &= \Big(\size(\ell) \ge 0 \land \size(r) \ge 0\Big) \land \Big(\height(\ell) = \height(r)\Big) \land \Big(\forall \lambda \in \ell, \rho \in r : \lambda \le a \le \rho \Big)\,,\\
g(t) &= \begin{multlined}[t] \Big(\size(\ell) \ge 0 \land \size(m) \ge 0 \land \size(r) \ge 0\Big) \land \Big(\height(\ell) = \height(m) = \height(r)\Big) \land {}\\ \Big(\forall \lambda \in \ell, \mu \in m, \rho \in r : \lambda \le a \le \mu \le b \le \rho \Big)\,.
\end{multlined}
\end{align}

L'invariant de représentation est alors donné par
\begin{equation}
\boxed{\ok(t) \equiv \size(t) = 0 \lor \Big(\type(t) = 2 \land f(t)\Big) \lor  \Big(\type(t) = 3 \land g(t)\Big)\,.}
\end{equation}

\section{Fonction d'abstraction}
Définissons une séquence d'entiers \(\seq{\cdot}\) par la règle de concaténation suivante:
\begin{align}
\seq{\seq{a}, \seq{b}} = \seq{\mathrm{concat}(a, b)}\,,
\end{align}
où \(a\) et \(b\) sont des séquences potentiellement vides ou des nombres
et \(\mathrm{concat}\) représente la concaténation des éléments de \(a\) et \(b\) (par exemple, si \(a = \seq{1, 2}\) et \(b = \seq{5}\), alors \(\mathrm{concat}(a, b) = \seq{1, 2, 5}\)).
En utilisant la même notation que pour l'invariant de représentation,
La fonction d'abstraction \(\abs(t)\) est alors donnée par
\begin{equation}
\boxed{\abs(t) =
\begin{cases}
\seq{}\,, & \textnormal{si } t = \textnormal{\inlinedafny|Vide|}\,,\\
\seq{\abs(\ell), a, \abs(r)}\,, & \textnormal{si } t = \textnormal{\inlinedafny|Deux|}(\ell, a, r)\,,\\
\seq{\abs(\ell), a, \abs(m), b, \abs(r)}\,, & \textnormal{si } t = \textnormal{\inlinedafny|Trois|}(\ell, a, m, b, r)\,.
\end{cases}}
\end{equation}

\section{Spécifications formelles}
\subsection{Spécification de \inlinedafny|insert(t: Tree, i: int) returns (t': Tree)|}
\subsubsection{Précondition}
La précondition est simplement que l'invariant de représentation soit respecté.
\begin{equation}
\boxed{\textnormal{Pre: } \ok(t)\,.}
\end{equation}

\subsubsection{Postcondition}
Comme la fonction \inlinedafny|insert| doit conserver le rep-invariant, il fait également partie de la postcondition.
En plus de cela, on requiert également que le multiensemble des éléments du nouvel arbre soit égal au multiensemble correspondant à l'arbre résultant de l'union de l'arbre original et de \(i\), où les multiensembles sont définis comme dans~\cite{blizard1991}.
\begin{equation}
\boxed{\textnormal{Post: } \ok(t') \land \elements(t') = \elements(t) \cup \{\!\!\{i\}\!\!\}\,.}
\end{equation}
Ici, \(\elements\) est un opérateur qui prend en argument un arbre 2-3 et retourne le multiensemble de ses éléments, alors que la notation \(\{\!\!\{\cdot\}\!\!\}\) dénote un multiensemble.

\subsection{Spécification de \inlinedafny|join(t1: Tree, t2: Tree) returns (t: Tree)|}
\subsubsection{Précondition}
La fonction \inlinedafny|join| a comme précondition que les deux arbres pris en argument soient des arbres 2-3 valables.
\begin{equation}
\boxed{\textnormal{Pre: } \ok(t1) \land \ok(t2)\,.}
\end{equation}

\subsubsection{Postcondition}
Le résultat de la fonction \inlinedafny|join| doit également être un arbre 2-3 valable.
En plus de cela, on requiert également que le multiensemble des éléments du nouvel arbre soit égal au multiensemble résultant de l'union des deux arbres initiaux.
\begin{equation}
\boxed{\textnormal{Post: } \ok(t) \land \elements(t) = \elements(t1) \cup \elements(t2)\,.}
\end{equation}
La fonction \(\elements\) est la même que pour \inlinedafny|insert|.

\section{Code}

\lstfile{../src/code.dfy}

\section*{Références}

\bibliography{ref}
\bibliographystyle{elsarticle-harv}\biboptions{authoryear}

\end{document}
\endinput