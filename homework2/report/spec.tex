\documentclass{elsarticle}
\usepackage[a4paper,left=2.5cm,right=1.5cm,top=1.5cm,bottom=1.5cm]{geometry}
\usepackage{natbib}
\usepackage{amsmath,amssymb,amsfonts,amsthm}
\usepackage{mathtools}
\usepackage[french]{babel}
\usepackage{bm}
\usepackage{algorithmic}
\usepackage{graphicx}
\usepackage{textcomp}
\usepackage{xcolor}
\usepackage{hyperref}
\usepackage{float}
\usepackage[T1]{fontenc}
\usepackage[utf8]{inputenc}
\usepackage{subcaption}
\usepackage{listings}
\usepackage{../../dafny}
\lstset{language=dafny}
\graphicspath{{img/}}
\usepackage{svg}

\makeatletter
\def\ps@pprintTitle{%
	\let\@oddhead\@empty
	\let\@evenhead\@empty
	\def\@oddfoot{\centerline{\thepage}}%
	\let\@evenfoot\@oddfoot}
\makeatother

\makeatletter
\def\blfootnote{\gdef\@thefnmark{}\@footnotetext}
\makeatother

\def\BibTeX{{\rm B\kern-.05em{\sc i\kern-.025em b}\kern-.08em
		T\kern-.1667em\lower.7ex\hbox{E}\kern-.125emX}}
\newcommand{\abs}[1]{\left\lvert#1\right\lvert}
\usepackage{siunitx}
\begin{document}
\title{Méthodes de conception de programmes \\ Devoir 2: 1, 2, 3\ldots{} Arbres!}
\date{12 mars 2019}

\address[add1]{École Polytechnique, Université catholique de Louvain, Place de l'Université 1, 1348 Ottignies-Louvain-la-Neuve, Belgique}

\author[add1]{Alexandre \textsc{Gobeaux}}
\ead{alexandre.gobeaux@student.uclouvain.be}

\author[add1]{Louis \textsc{Navarre}}
\ead{navarre.louis@student.uclouvain.be}

\author[add1]{Gilles \textsc{Peiffer}}
\ead{gilles.peiffer@student.uclouvain.be}

\begin{abstract}
Ce papier donne les invariants de représentation, la fonction d'abstraction et les spécifications des fonctions \inlinedafny|insert| et \inlinedafny|join| pour une implémentation des arbres 2-3 basée sur \cite{algs4th}.
\end{abstract}
\maketitle

\section{Description du problème et de la solution}

\section*{Références}

\bibliography{ref}
\bibliographystyle{elsarticle-harv}\biboptions{authoryear}

\end{document}
\endinput