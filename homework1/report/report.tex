\documentclass{elsarticle}
\usepackage[a4paper,left=2.25cm,right=1.5cm,top=1.5cm,bottom=1.5cm]{geometry}
\usepackage{cite}
\usepackage{amsmath,amssymb,amsfonts}
\usepackage{mathtools}
\usepackage[french]{babel}
\usepackage{algorithmic}
\usepackage{graphicx}
\usepackage{textcomp}
\usepackage{xcolor}
\usepackage{hyperref}
\usepackage{float}
\usepackage[T1]{fontenc}
\usepackage[utf8]{inputenc}
\usepackage{subcaption}
\usepackage{listings}
\usepackage{dafny}
\lstset{language=dafny}
\graphicspath{{img/}}
\usepackage{svg}

\makeatletter
\def\ps@pprintTitle{%
	\let\@oddhead\@empty
	\let\@evenhead\@empty
	\def\@oddfoot{\centerline{\thepage}}%
	\let\@evenfoot\@oddfoot}
\makeatother

\def\BibTeX{{\rm B\kern-.05em{\sc i\kern-.025em b}\kern-.08em
		T\kern-.1667em\lower.7ex\hbox{E}\kern-.125emX}}
\newcommand{\abs}[1]{\left\lvert#1\right\lvert}
\usepackage{siunitx}
\begin{document}
\title{Méthodes de conception de programmes \\ Devoir 1: Preuve de programme}
\date{12 mars 2019}

\address[add1]{École Polytechnique, Université catholique de Louvain, Place de l'Université 1, 1348 Ottignies-Louvain-la-Neuve, Belgique}

\author[add1]{Alexandre \textsc{Gobeaux}}
\ead{alexandre.gobeaux@student.uclouvain.be}
\author[add1]{Louis \textsc{Navarre}}
\ead{navarre.louis@student.uclouvain.be}
\author[add1]{Gilles \textsc{Peiffer}}
\ead{gilles.peiffer@student.uclouvain.be}

\begin{abstract}
Ce document donne un algorithme pour résoudre le problème 2SUM ainsi qu'une preuve de correction totale pour celui-ci, par rapport aux spécifications définies.
\end{abstract}
\maketitle

\section{Description du problème et de la solution}
Le problème à résoudre est celui d'une séquence d'entiers triée $a$ dans laquelle il faut déterminer si oui ou non deux entrées définies par ces entiers (potentiellement deux fois les mêmes) ont une somme égale à un entier prédéfini $s$.
Si oui, alors le programme doit renvoyer \og \inlinedafny{true} \fg, ainsi que les indices des deux entiers qui satisfont la condition, sinon il renvoie \og \inlinedafny{false} \fg, et la valeur des indices n'a pas d'importance.

Notre algorithme fonctionne en temps linéaire ($\mathcal{O}(n)$), en utilisant deux pointeurs:
le premier commence au premier élément du tableau, alors que le deuxième commence à la fin.
Tant que le premier pointeur est plus petit ou égal au second, on évalue leur somme à chaque itération:
\begin{itemize}
	\item si celle-ci est plus petite que $s$, on avance d'une unité le premier pointeur;
	\item si celle-ci est plus grande que $s$, on recule d'une unité le deuxième pointeur et
	\item si celle-ci est égale à $s$, on retourne la valeur des indices ainsi que la valeur \og \inlinedafny{true}\fg.
\end{itemize}
L'algorithme peut donc se terminer de deux façons: soit il trouve une paire d'indices qui satisfait la condition, passant donc dans la troisième possibilité ci-dessus,
soit les pointeurs se croisent, ce qui signifie que la séquence ne contient pas de paire satisfaisant la condition.

\section{Spécification formelle}
\subsection{Précondition}
La précondition est que la séquence soit triée.
Formellement, on demande que
\[
\forall k, \ell \mid 0 \le k \le \ell < \abs{a} :: a[k] \le a[l]\,.
\]
\subsection{Modifie}
L'algorithme ne modifie rien.
\subsection{Postcondition}
La première postcondition est, informellement, que si la valeur booléenne est vraie,
alors $i$ et $j$ contiennent les bonnes valeurs d'indices.
La seconde postcondition est que la valeur booléenne est fausse si et seulement si il n'existe aucune paire d'indices telle que la somme des entrées de $a$ définies par cette paire d'indice n'est égale à $s$; nous pouvons aussi dire que, pour toute paire d'indices dans le domaine de $a$, la somme des entrées définies par ces indices est différente de $s$.
Formellement (en appelant $\texttt{found}$ la valeur booléenne),
\begin{align*}
&\texttt{found} \implies (0 \le i \le j < \abs{a}) \land (a[i] + a[j] = s)\,, \\
\land \lnot &\texttt{found} \iff \forall m, n \mid 0 \le m \le n < \abs{a} :: a[m] + a[n] \ne s\,.
\end{align*}

\section{Implémentation}
L'implémentation de l'algorithme donné précédemment en Dafny est la suivante.
\lstfile{../src/find_sum.dfy}

\section{Graphe d'exécution}
Le graphe d'exécution pour l'algorithme ci-dessus est donné sur la \textsc{Figure}~\ref{fig:execgraph}.
\begin{figure}[htbp]
	\centering
	\includesvg[width=\textwidth]{execgraph}
	\caption{Le graphe d'exécution complet pour le programme donné.}
	\label{fig:execgraph}
\end{figure}
Afin de faciliter les preuves pour les différents chemins simples,
la \textsc{Figure}~\ref{fig:simplepaths} reprend ceux-ci de façon plus lisible.
\begin{figure}[htbp]
\centering
	\begin{subfigure}{0.3\textwidth}
	\centering
	\includesvg[width=0.4\textwidth]{simplepath1}
	\label{fig:1}
	\end{subfigure}\hspace{3cm}
	\begin{subfigure}{0.3\textwidth}
	\centering
	\includesvg[width=0.5\textwidth]{simplepath2}
	\label{fig:2}
	\end{subfigure}\\
	\begin{subfigure}{0.3\textwidth}
	\centering
	\includesvg[width=0.5\textwidth]{simplepath3}
	\label{fig:3}
	\end{subfigure}\hfill
	\begin{subfigure}{0.3\textwidth}
	\centering
	\includesvg[width=0.5\textwidth]{simplepath4}
	\label{fig:4}
	\end{subfigure}\hfill
	\begin{subfigure}{0.3\textwidth}
	\centering
	\includesvg[width=0.5\textwidth]{simplepath5}
	\label{fig:5}
	\end{subfigure}
	\caption{Représentations plus propres des différents chemins simples.}
	\label{fig:simplepaths}
\end{figure}

\begin{thebibliography}{00}
	\bibitem{robtex} \url{https://www.robtex.com/dns-lookup/www.aliexpress.com}, accessed on December 1, 2018.
\end{thebibliography}
\end{document}
